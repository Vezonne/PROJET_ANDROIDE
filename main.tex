\documentclass{article}
\usepackage[utf8]{inputenc}
\usepackage{amsmath}




\title{Outil d'appariement pour l'attribution des projets Androide}



\author{
  Perotti-Valle Rayan \\
  \and
  Savarit Felix \\
  \and
  Thomasson Malo
}
\date{\today}

\begin{document}

\maketitle

\newpage

\renewcommand{\contentsname}{Sommaire} % Change the name of the table of contents

\tableofcontents{} % Sommaire

\newpage

\section{Définition du problème}
Ce projet à pour but de recuperer les preferences des etudiants pour leur choix des projets Androide dans le cadre de leur UE de projet.
Dans un premier temps nous récolterons tous les choix des étudiants et professeurs à l'aide d'une interface web liée à une base de données. Cela nous donnera une liste. 
Dans un second temps nous utiliserons un programme linéaire sur les listes de preferences des étudiants et professeurs pour maximiser le nombre d'étudiants affectés à un projet tout en maximisant la somme des \textbf{scores} des etudiants et des \textbf{scores} des encadrants. On appelle le \textbf{score} d'un etudiant (resp encadrant) la position de son projet (resp groupe) dans sa liste de preferences, le projet (resp groupe) preferé d'un étudiant (resp encadrant) est n avec n son nombre de projet (resp groupe) dans sa liste de préférence. Son dernier projet (resp groupe) préféré a donc un score de 1.

\section{Objectifs}


\subsection{Methode de résolution}


\textbf{Programme mathematiques :}

Variables : \[ x_{eP} \] 
\[ x_{GP} \]
\begin{itemize}
\item P projet
\item G groupe
\item e étudiant
\item k nombre cible d'étudiants affectés
\item score(G,P) : la somme du score pondérée des étudiants du groupe G et du score de l'enseignant du projet P.  
\end{itemize}

Maximiser la somme de tous les scores :

\[ \text{Maximiser} \sum_{G} \sum_{P} x_{GP} \cdot \text{score}(G, P) \]

\textbf{Contraintes :}

1. Il faut au moins k étudiant affectés à un projet : 

\[ \sum_{e} \sum_{P} x_{eP} \geq k \]

3. Chaque groupe d'étudiants est affecté à au plus un projet :

\[ \quad \forall G \sum_{P} x_{GP} \leq 1 \]

4. Chaque projet doit etre affecté à au plus 1 groupe : 

\[\quad \forall P \sum_{G} x_{GP} \leq 1 \]

5. un étudiant ne doit pas avoir plusieurs projet : 

\[\quad \forall P, \forall e \in G, x_{GP} = x_{eP} \]

6. Contrainte de stabilité 

\vspace{1em}

Il ne faut pas qu'un groupe A soit disponible ou la majorité de A ne préfére pas le projet P qui leur 
a été attribué et que la majorité de A soit individuellement préféré dans un autre projet qu'il 
prefere à P par leur encadrant respectifs ou que ces projets soit disponibles et que les groupes avec 
lesquels ils ont postulés préférent ces projets auxquels ils ont été attribué ou que ces groupes soit 
disponibles

\vspace{1em}

7. Les variables de décision sont binaires :

\[ x_{eP} \in \{0, 1\} \]
\[ x_{GP} \in \{0, 1\} \]

\subsection{Outils de recueil des voeux etudiants/encadrants}

On developpera une interface web en javascript/html/CSS qui aura un coté recueil des voeux étudiant et un coté recueil des voeux encadrant.
Le but étant de recueillir les preferences de chacun dans une base de données. Une fois les données recuperées on fait tourner l'algo dessus en local.
On pourra si necessaire ajouter des fonctionnalités supplementaires comme le depot de CV ou lettre de motivation sur demande des encadrants.


\section{Délais}

Semaine du 20 Mai

\end{document}
